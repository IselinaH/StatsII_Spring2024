\documentclass[12pt,letterpaper]{article}
\usepackage{graphicx,textcomp}
\usepackage{natbib}
\usepackage{setspace}
\usepackage{fullpage}
\usepackage{color}
\usepackage[reqno]{amsmath}
\usepackage{amsthm}
\usepackage{fancyvrb}
\usepackage{amssymb,enumerate}
\usepackage[all]{xy}
\usepackage{endnotes}
\usepackage{lscape}
\newtheorem{com}{Comment}
\usepackage{float}
\usepackage{hyperref}
\newtheorem{lem} {Lemma}
\newtheorem{prop}{Proposition}
\newtheorem{thm}{Theorem}
\newtheorem{defn}{Definition}
\newtheorem{cor}{Corollary}
\newtheorem{obs}{Observation}
\usepackage[compact]{titlesec}
\usepackage{dcolumn}
\usepackage{tikz}
\usetikzlibrary{arrows}
\usepackage{multirow}
\usepackage{xcolor}
\newcolumntype{.}{D{.}{.}{-1}}
\newcolumntype{d}[1]{D{.}{.}{#1}}
\definecolor{light-gray}{gray}{0.65}
\usepackage{url}
\usepackage{listings}
\usepackage{color}

\definecolor{codegreen}{rgb}{0,0.6,0}
\definecolor{codegray}{rgb}{0.5,0.5,0.5}
\definecolor{codepurple}{rgb}{0.58,0,0.82}
\definecolor{backcolour}{rgb}{0.95,0.95,0.92}

\lstdefinestyle{mystyle}{
	backgroundcolor=\color{backcolour},   
	commentstyle=\color{codegreen},
	keywordstyle=\color{magenta},
	numberstyle=\tiny\color{codegray},
	stringstyle=\color{codepurple},
	basicstyle=\footnotesize,
	breakatwhitespace=false,         
	breaklines=true,                 
	captionpos=b,                    
	keepspaces=true,                 
	numbers=left,                    
	numbersep=5pt,                  
	showspaces=false,                
	showstringspaces=false,
	showtabs=false,                  
	tabsize=2
}
\lstset{style=mystyle}
\newcommand{\Sref}[1]{Section~\ref{#1}}
\newtheorem{hyp}{Hypothesis}

\title{Problem Set 4}
\date{Due: April 12, 2024}
\author{Applied Stats II}


\begin{document}
	\maketitle
	\section*{Instructions}
	\begin{itemize}
	\item Please show your work! You may lose points by simply writing in the answer. If the problem requires you to execute commands in \texttt{R}, please include the code you used to get your answers. Please also include the \texttt{.R} file that contains your code. If you are not sure if work needs to be shown for a particular problem, please ask.
	\item Your homework should be submitted electronically on GitHub in \texttt{.pdf} form.
	\item This problem set is due before 23:59 on Friday April 12, 2024. No late assignments will be accepted.

	\end{itemize}

	\vspace{.25cm}
\section*{Question 1}
\vspace{.25cm}
\noindent We're interested in modeling the historical causes of child mortality. We have data from 26855 children born in Skellefteå, Sweden from 1850 to 1884. Using the "child" dataset in the \texttt{eha} library, fit a Cox Proportional Hazard model using mother's age and infant's gender as covariates. Present and interpret the output.

\section*{Answer Question 1}
\lstinputlisting[language=R]{/Users/iseli/Documents/GitHub/StatsII_Spring2024/problemSets/PS04/my_answers/PS4.R}

\section*{Results of Cox Proportional Hazards Model}

The Cox Proportional Hazards model was fitted using mother's age and infant's gender as covariates. The dataset consisted of 26,574 observations with 5,616 events. Below are the detailed results of the model:

\begin{table}[H]
	\centering
	\begin{tabular}{lccccc}
		\hline
		Variable & Coef. & Exp(Coef.) & SE(Coef.) & Z & Pr(>|z|) \\
		\hline
		m.age & 0.007617 & 1.007646 & 0.002128 & 3.580 & 0.000344 *** \\
		sexfemale & -0.082215 & 0.921074 & 0.026743 & -3.074 & 0.002110 ** \\
		\hline
	\end{tabular}
	\caption{Coefficients and hazard ratios from the Cox model.}
\end{table}

Significance codes:  0 ‘***’ 0.001 ‘**’ 0.01 ‘*’ 0.05 ‘.’ 0.1 ‘ ’ 1

\begin{table}[H]
	\centering
	\begin{tabular}{lcccc}
		\hline
		Variable & exp(coef) & exp(-coef) & lower .95 & upper .95 \\
		\hline
		m.age & 1.0076 & 0.9924 & 1.003 & 1.0119 \\
		sexfemale & 0.9211 & 1.0857 & 0.874 & 0.9706 \\
		\hline
	\end{tabular}
	\caption{Hazard ratios and 95\% confidence intervals for Cox model covariates.}
\end{table}

Model Fit Statistics:

\begin{itemize}
	\item Concordance: 0.519  (SE = 0.004)
	\item Likelihood ratio test: 22.52 on 2 df, p=1e-05
	\item Wald test: 22.52 on 2 df, p=1e-05
	\item Score (logrank) test: 22.53 on 2 df, p=1e-05
\end{itemize}
\end{document}
