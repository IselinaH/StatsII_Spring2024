\documentclass[12pt,letterpaper]{article}
\usepackage{graphicx,textcomp}
\usepackage{natbib}
\usepackage{setspace}
\usepackage{fullpage}
\usepackage{color}
\usepackage[reqno]{amsmath}
\usepackage{amsthm}
\usepackage{fancyvrb}
\usepackage{amssymb,enumerate}
\usepackage[all]{xy}
\usepackage{endnotes}
\usepackage{lscape}
\newtheorem{com}{Comment}
\usepackage{float}
\usepackage{hyperref}
\newtheorem{lem} {Lemma}
\newtheorem{prop}{Proposition}
\newtheorem{thm}{Theorem}
\newtheorem{defn}{Definition}
\newtheorem{cor}{Corollary}
\newtheorem{obs}{Observation}
\usepackage[compact]{titlesec}
\usepackage{dcolumn}
\usepackage{tikz}
\usetikzlibrary{arrows}
\usepackage{multirow}
\usepackage{xcolor}
\newcolumntype{.}{D{.}{.}{-1}}
\newcolumntype{d}[1]{D{.}{.}{#1}}
\definecolor{light-gray}{gray}{0.65}
\usepackage{url}
\usepackage{listings}
\usepackage{color}

\definecolor{codegreen}{rgb}{0,0.6,0}
\definecolor{codegray}{rgb}{0.5,0.5,0.5}
\definecolor{codepurple}{rgb}{0.58,0,0.82}
\definecolor{backcolour}{rgb}{0.95,0.95,0.92}

\lstdefinestyle{mystyle}{
	backgroundcolor=\color{backcolour},   
	commentstyle=\color{codegreen},
	keywordstyle=\color{magenta},
	numberstyle=\tiny\color{codegray},
	stringstyle=\color{codepurple},
	basicstyle=\footnotesize,
	breakatwhitespace=false,         
	breaklines=true,                 
	captionpos=b,                    
	keepspaces=true,                 
	numbers=left,                    
	numbersep=5pt,                  
	showspaces=false,                
	showstringspaces=false,
	showtabs=false,                  
	tabsize=2
}
\lstset{style=mystyle}
\newcommand{\Sref}[1]{Section~\ref{#1}}
\newtheorem{hyp}{Hypothesis}

\title{Assignment 1}
\date{Due: March 29, 2024}
\author{Social Forecasting}


\begin{document}
	\maketitle
	\section*{Instructions}
	\begin{itemize}
	\item 
	When submitting your answers, please 1) show your code and 2) make sure your presentation is clear and easy to follow. Rmarkdown will make it easy to produce a suitable report, but you are welcome to use another approach if you prefer.
	\end{itemize}

	\vspace{.25cm}
\section*{Answer Task 1}
\vspace{.25cm}

\begin{enumerate}
	
		\lstinputlisting[language=R, firstline=1,lastline=13]{/Users/iseli/Documents/Social Forecasting/Iselina Hernandez Assignment 1.R} 
\end{enumerate}
		
		\vspace{.25cm}
	\section*{Answer Task 2}
	\vspace{.25cm}
	
	\begin{enumerate}
		
		\lstinputlisting[language=R, firstline=16,lastline=20]{/Users/iseli/Documents/Social Forecasting/Iselina Hernandez Assignment 1.R} 
	\end{enumerate}	
	
		\vspace{.25cm}
\section*{Answer Task 3}
\vspace{.25cm}

\begin{enumerate}
	\lstinputlisting[language=R, firstline=23,lastline=31]{/Users/iseli/Documents/Social Forecasting/Iselina Hernandez Assignment 1.R} 
\end{enumerate}		
	
	\section{Forecasts vs Actual Data}
	\begin{figure}[ht]
		\centering
		\includegraphics[width=0.8\textwidth]{/Users/iseli/Documents/Social Forecasting/Rplot 1.pdf}
		\caption{Forecasts vs Actual Data (2016-2021)}
		\label{fig:forecasts}
	\end{figure}

		\vspace{.25cm}
\section*{Answer Task 4}
\vspace{.25cm}

\begin{enumerate}
	\lstinputlisting[language=R, firstline=34,lastline=93]{/Users/iseli/Documents/Social Forecasting/Iselina Hernandez Assignment 1.R} 
\end{enumerate}		

		\vspace{.25cm}
\section*{Answer Task 5}
\vspace{.25cm}

\begin{enumerate}
	\lstinputlisting[language=R, firstline=96,lastline=126]{/Users/iseli/Documents/Social Forecasting/Iselina Hernandez Assignment 1.R} 
\end{enumerate}		
	
	\section{Forecasts vs Actual Data}
	\begin{figure}[ht]
		\centering
		\includegraphics[width=\textwidth]{/Users/iseli/Documents/Social Forecasting/Rplot.pdf}
		\caption{Forecasts vs Actual Data (2016-2021)}
		\label{fig:forecasts}
	\end{figure}
	
	As shown in Figure \ref{fig:forecasts}, the drift method provides the closest approximation to the actual data from 2016 to 2021.	
	
		\vspace{.25cm}
\section*{Answer Task 6}
\vspace{.25cm}

\begin{enumerate}
	\lstinputlisting[language=R, firstline=129,lastline=130]{/Users/iseli/Documents/Social Forecasting/Iselina Hernandez Assignment 1.R} 
\end{enumerate}		

\end{enumerate}
\end{document}
